\chapter{The Computer} % (fold)
\label{chap:The Computer}

\section{Process}

The Author has decided to opt for a requirement based, test driven development approach. Thus the current plan for the development is the following:
\begin{enumerate}
  \item Define loose architecture. 
  \item For each component of the architecture define (testable) requirements. 
  \item Justify requirements
  \item For each component:
  \begin{itemize}
    \item Write testcases
    \item Write corresponding verilog code.
  \end{itemize}
\end{enumerate}

\subsection{Requirements}
The requirements can be loosely split into three categories:

\begin{itemize}
  \item Turing requirements
  \item Architechtural requirements
  \item Feature requirements
\end{itemize}


\newtheorem{turing-requirement}{Turing Req.}[subsection]
Turing requirements are requirements to the architectures that exist to give it turing completeness. Thus these requirements will always be derived from citation lol (Turing Machine): 
\begin{itemize}
  \item Infinite memory
  \item Ability to modify memory
  \item Ability to conditionally execute
\end{itemize}

\newtheorem{arch-requirement}{Arch. Req.}[subsection]
Architecturally required are conecpts that must be implemented to create interoperability between the components. They are required to be implemented under every circumstance given a certrain architecture. 

\newtheorem{feat-requirement}{Feat. Req.}[subsection]
Feature Requirements: For now, they exist.


\section{Architecture}

\subsection{Arithmetic Logic Unit}

\begin{turing-requirement}
  Ability to add two data words.
\end{turing-requirement}

\begin{turing-requirement}
  Generation of at least one status flag (overflow, underflow and/or remainder of divison). 
\end{turing-requirement}

\begin{arch-requirement}
  Ability to take in data from 2 registers.
\end{arch-requirement}

\begin{arch-requirement}
  Ability to output data to the bus. 
\end{arch-requirement}

\begin{arch-requirement}
  Ability to execute all calculations within one timing state.
\end{arch-requirement}

\begin{feat-requirement}
  Ability to subtract two data words.
\end{feat-requirement}

\begin{feat-requirement}
  Ability to multiply two data words.
\end{feat-requirement}

\begin{feat-requirement}
  Ability to divide two data words.
\end{feat-requirement}

\begin{feat-requirement}
  Ability to shift a data word bitwise.
\end{feat-requirement}

\begin{feat-requirement}
  Ability to shift a data word bitwise.
\end{feat-requirement}

\begin{feat-requirement}
  Ability to rotate a data word bitwise.
\end{feat-requirement}

\subsection{Memory}

\begin

\subsection{Registers}

\subsection{Control Unit}

\subsection{Micro programming}

\subsection{Macro Programming}

\subsection{Input/Output}

% chapter The Computer (end)

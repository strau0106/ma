\chapter{The Computer} % (fold)
\label{chap:The Computer}

\section{Process}

The Author has decided to opt for a requirement based, test driven development approach. Thus the current plan for the development is the following:
\begin{enumerate}
  \item Define loose architecture. 
  \item For each component of the architecture define (testable) requirements. 
  \item Justify requirements
  \item For each component:
  \begin{itemize}
    \item Write testcases
    \item Write corresponding verilog code.
  \end{itemize}
\end{enumerate}

\section{Architecture}
\newtheorem{requirement}{Req.}[subsection]

\subsection{Arithmetic Logic Unit}


\begin{requirement}
  Ability to take in data from 2 registers.
\end{requirement}

\begin{requirement}
  Ability to output data to the bus. 
\end{requirement}

\begin{requirement}
  Ability to add two data words.
\end{requirement}

\begin{requirement}
  Ability to subtract two data words.
\end{requirement}

\begin{requirement}
  Ability to multiply two data words.
\end{requirement}

\begin{requirement}
  Ability to divide two data words.
\end{requirement}

\begin{requirement}
  Ability to shift a data word bitwise.
\end{requirement}

\begin{requirement}
  Ability to shift a data word bitwise.
\end{requirement}

\begin{requirement}
  Ability to rotate a data word bitwise.
\end{requirement}

\begin{requirement}
  Ability to execute all calculations within one timing state.
\end{requirement}

\begin{requirement}
  Generation of status flags for: overflow, underflow and remainder of divison. 
\end{requirement}



% chapter The Computer (end)

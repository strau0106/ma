\subsection{Registers}

Registers are technically speaking redundant, as the computer already has a form of memory. In practice regs still exist: 
- Regs are in silicon accessible faster than memory. 
- Storage acccess in case of this specific arch is simpler w/o relative addressing etc. 
- ALU needs access to what it calculates, thus registers, unless directly connected to memory, which in practice is not possible (memory seperate silicon from chip). 
Thus registers are considered feature requirements

\begin{feat-requirement}
  Ability to store a data word.
\end{feat-requirement}

\begin{feat-requirement}
  Ability to output a data word.
\end{feat-requirement}

\begin{turing-requirement}
  Ability to load a data word for storage.
\end{turing-requirement}

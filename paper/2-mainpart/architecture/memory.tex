\subsection{Memory}

\begin{turing-requirement}
Must allow writing to and reading from a memory location
\end{turing-requirement}

As instructions need data to operate on, the data, operands, and instructions should ideally be stored in a way that allows accessing one together with the other. The simplest approach would be to transfer the instruction together with the operand on to the bus, as in the SAP-2 architecture \cite{malvino1983a}.

As an 8-bit bus width is required, equally splitting the bus into 4 bits for the instruction and 4 bits for the operand would be the most straightforward approach.

This however results in only 16 possible instructions and the highest possible operand value of 15. Although this would not impact turing completeness or any other requirement, it does not allow for an easily usable instruction set. In fact, manual programming of any "complex" task would be very difficult. One could achieve higher instruction count by reserving one instruction for all non-operand instructions and then using the last 4 bits to identify the individual instruction. This would allow for 31 instructions, 15 of which with operands up to 15.

Increasing the size of the instruction or operand is no option either as the other component would need to shrink, resulting in the architecture either having less instructions or smaller operands.

A solution to this problem is the introduction of a second data word that is stored in memory alongside, essentially increasing the address bus width by one but instead but running this wire as part of the control bus. The indication of which data word is thus tied to the instruction's microprogramming and is not intended to be chosen at runtime. Additionally this way of increasing the address bus does not require running a wire to every component, keeping the physical layout simple. Finally the size of the stored programm is also smaller, as the indication of the used data word is only stored once in the microcode contrary to repetitvely storing it in the program code.  

As the active data word must be configured per microinstruction, before computer execution specifying each data word, each data word is assigned a purpose. The first one storing the instruction and the second storing the operand.    

\begin{feat-requirement}
Must store, for each memory address, store two data words. A control signal shall indicate which data word to access. The first data word shall be intended for instructions and the second for operands.
\end{feat-requirement}

While an 8 bit operand is considered sufficient for this architecture a program with only 255 instruction steps is not. Furthermore, to make the architecture Turing complete the architecture needs to be able to address infinite memory. To realize this, the primary addressing mode shall be relative. Additionally, the address width of the memory shall be extendable beyond the width of the 8-bit bus.

\begin{feat-requirement}
Must retrieve data with a relative memory address. 
\end{feat-requirement}

To allow for relative addressing, the program counter (PC) must be stored in full length. Additionally, to allow useful programming a second general purpose memory address register (MAR) shall be implemented. % TODO: Explain better

\begin{feat-requirement}
Must store the program counter and memory address register.
\end{feat-requirement}

For certain features, such as a call stack, it is nonetheless useful to allow for abosulte addressing. Additionally, first $8$-bit of the PC and MAR shall be overwriteable for absolute access but with an offset. 

\begin{feat-requirement}
Must retrieve data with an absolute memory address. 
\end{feat-requirement}


When the program is executed the first micro instruction will always be to increment the PC to get the next instruction. As that one cannot come from the bus, as no instruction or memory is present, the PC must be able to be incremented without the bus.
\begin{feat-requirement}
    Must be able to increment the program counter by one with a specific control signal.
\end{feat-requirement}

\iffalse
With relative jumps of only 255 but bigger word counts. 
Assembler could impl like jump over 

PSEUDO
NORMAL PROGRAM
JMP2
JPD 255 
NORMAL PROGRAM RESUMES HERE
.
.
.
SOME OTHER PROGRAM THAT NEEDS TO JUMP MORE THAN ONE

\fi



% \lstinputlisting[language=verilog]{../computer/src/modules/memory.sv}

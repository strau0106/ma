\subsection{Arithmetic Logic Unit}

\begin{turing-requirement}
  Ability to add two data words.
\end{turing-requirement}

\begin{turing-requirement}
  Generation of at least one status flag (overflow, underflow and/or remainder of divison). 
\end{turing-requirement}

\begin{arch-requirement}
  Ability to take in data from 2 registers.
\end{arch-requirement}

\begin{arch-requirement}
  Ability to output data to the bus. 
\end{arch-requirement}

\begin{arch-requirement}
  Ability to execute all calculations within one timing state.
\end{arch-requirement}

\begin{feat-requirement}
  Ability to subtract two data words.
\end{feat-requirement}

\begin{feat-requirement}
  Ability to multiply two data words.
\end{feat-requirement}

\begin{feat-requirement}
  Ability to divide two data words.
\end{feat-requirement}

\begin{feat-requirement}
  Ability to shift a data word bitwise.
\end{feat-requirement}

\begin{feat-requirement}
  Ability to rotate a data word bitwise.
\end{feat-requirement}

\begin{feat-requirement}
  Must not exert undefined behaviour in case of undefined control signals. 
\end{feat-requirement}

\begin{feat-requirement}
  Must not output unless control signals indicate to do so. 
\end{feat-requirement}

\begin{feat-requirement}
  Ability to AND, OR and XOR two data words.
\end{feat-requirement}

\begin{feat-requirement}
  Ability to perform NOT on a data word.
\end{feat-requirement}


Given those requirements a list of signals going into and out of the module can be compiled. 

\begin{table}[]
\begin{tabular}{ccc}
Type& Name & Purpose \\ \hline
I   & Clock & Timinga \\
O   & Bus     & Data output         \\
I   & Register 1 and 2 & Data input \\
I   & ALU control word & Control \\
O   & Status flag word & Control
\end{tabular}
\caption{}
\label{tab:alu-i/o}
\end{table} 

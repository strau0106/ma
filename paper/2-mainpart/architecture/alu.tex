\subsection{Arithmetic Logic Unit}
The Arithmetic Logic Unit is the module performing all arithmetic and logical operations on operands. \cite{vonneuman1945a}. The ALU is the centerpiece of the computer, as it is the only module that modifies, or synthesizes, new data.
Generally the addition operation is regarded as the simplest operation and thus shall be the first operation to be implemented.

\begin{turing-requirement}
  Ability to add two data words.
\end{turing-requirement}

To extend the computers feature set subtraction, shift, rotation and bit shift operations shall be implemented. Multiplication and division shall not be implemented, as unlike multiplication and division, these mathematical operations are performable in a single iteration. 

Binary multiplication is normally achieved by repeated addition of the multiplicand to itself, whilst counting the addition operations, stopping when the multiplier is reached \cite{cit.needed}. The simplest way of performing a division, specifically integer division, is by repeated subtraction of the divisor from the dividend until a number smaller than the divisor is reached.
As the architecture implements repeatable operations based on conditions, multiplication and division can be performed in macro code. 

Likewise, rotation and shift operations shall only be implemented for a single bit, as any other shift or rotation can only be achieved by performing the shift multiple times or by having multiple shift and rotate circuits in series. 

\begin{feat-requirement}
  Ability to subtract two data words.
\end{feat-requirement}

\begin{feat-requirement}
  Ability to shift a data word bitwise.
\end{feat-requirement}

\begin{feat-requirement}
  Ability to rotate a data word bitwise.
\end{feat-requirement}

\begin{feat-requirement}
  Ability to AND, OR and XOR two data words.
\end{feat-requirement}

\begin{feat-requirement}
  Ability to NOT a data word.
\end{feat-requirement}

Furthermore, the computer is required to be able to execute conditionally, thus requiring a data point indicating if a condition is true or not. This indication is called a flag. The two most common flags are the zero flag and the carry flag, as they are easily generated mathematically. 

The carry flag, which indicates whether or not an addition in the ALU has overflown, so resulted in a number which is larger than the 8-bit data word size. Apart from being useful in conditional jumps, the carry flag can also be used to detect if a calculation has overflown, and thus if the result is incorrect.

\begin{turing-requirement}
  Generation of a carry flag. 
\end{turing-requirement}

The zero flag indicates if the result of an operation is zero. This is not only useful for looping a fixed number of times, but also for comparing two data words. As the ALU is designed to subtract two data words, a zero flag allows for easy determination if two data words are equal, by subtracting them from each other. 

\begin{feat-requirement}
  Generation of a zero flag.
\end{feat-requirement}

Given the fact that the architecture has an 8-bit data bus, and the ALU performing 8-bit operations, the ALU will always need to buffer the two operands. Instead of having the ALU buffer the operands, with two additional registers, the ALU shall access two registers directly, that are otherwise, indepent of the ALU.
\begin{arch-requirement}
  Ability to take in data from 2 registers.
\end{arch-requirement}

As the calculation results need to return to registers and memory, the ALU must have the ability to output data to the bus.
\begin{arch-requirement}
  Ability to output operation results to the bus. 
\end{arch-requirement}

Finally, the ALU shall not exert undefined behaviour in case of undefined control signals, and shall not output unless control signals indicate to do so. 

\begin{feat-requirement} \label{req:alu-undef-behavior}
  Must not exert undefined behaviours in case of undefined control signals. 
\end{feat-requirement}

\begin{feat-requirement} \label{req:alu-no-output}
  Must not output unless control signals indicated to do so. 
\end{feat-requirement}


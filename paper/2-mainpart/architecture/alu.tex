\subsection{Arithmetic Logic Unit}

\begin{turing-requirement}
  Ability to add two data words.
\end{turing-requirement}

\begin{turing-requirement}
  Generation of at least one status flag (overflow, underflow and/or remainder of divison). 
\end{turing-requirement}

\begin{arch-requirement}
  Ability to take in data from 2 registers.
\end{arch-requirement}

\begin{arch-requirement}
  Ability to output data to the bus. 
\end{arch-requirement}

\begin{arch-requirement}
  Ability to execute all calculations within one timing state.
\end{arch-requirement}

\begin{feat-requirement}
  Ability to subtract two data words.
\end{feat-requirement}

\begin{feat-requirement}
  Ability to multiply two data words.
\end{feat-requirement}

\begin{feat-requirement}
  Ability to divide two data words.
\end{feat-requirement}

\begin{feat-requirement}
  Ability to shift a data word bitwise.
\end{feat-requirement}

\begin{feat-requirement}
  Ability to rotate a data word bitwise.
\end{feat-requirement}

\begin{feat-requirement}
  Must not exert undefined behaviour in case of undefined control signals. 
\end{feat-requirement}

\begin{feat-requirement}
  Must not output unless control signals indicate to do so. 
\end{feat-requirement}

\begin{feat-requirement}
  Ability to AND, OR and XOR two data words.
\end{feat-requirement}

\begin{feat-requirement}
  Ability to perform NOT on a data word.
\end{feat-requirement}

Each OP as individual wire, because decoding instructions is not the purpose of the ALU, although this means, that it needs to be checked if it is actually only one OP. (This is normal, vgl. Digital Computer electronics p. 175/ von Neumann). 

Also decreased complexity 9 vs. 4 wires and a bunch of decoding logic? Not limited in package pins etc. 

Direct Register Access because wanting to save timing states and is more easier. 

Given those requirements a list of signals going into and out of the module can be compiled. 

\begin{table}[]
\begin{tabular}{ccc}
Type& Name & Purpose \\ \hline
I   & Clock & Timinga \\
O   & Bus     & Data output         \\
I   & Register A and B & Data input \\
I   & ALU control word & Control \\
O   & Status flag word & Control
\end{tabular}
\caption{}
\label{tab:alu-i/o}
\end{table}

Given this Netlist

Karnaugh map of Operations. 


\lstinputlisting[language=verilog]{../computer/src/modules/alu.sv}

op sum to check if only one op is started

check if out to make sure to only output if said to do so. (three state output)

Each indivdual operation is performed if given.

Only one bit rotations possible since there is no length that can be put in, since only one, the second data word, value can be given in. Additionally, the implementation is much easier, if just one bit...

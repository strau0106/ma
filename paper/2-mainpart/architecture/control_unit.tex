\subsection{Control Unit}
Following the VNA the module orchestrating the interaction and operation of all modules is called the Control Unit. It is responsible for the generation of the control word, based on the macro program, the current state of the computer and current timing information. For every given combination of this input the control word must produce a specific output. In historic computer architectures this was achieved by carefully creating a net of logic gates, which would produce the correct output for every input. With the increasing complexity and the fact that once physically produced this control logic could not be changed, the concept of the microcode was invented \cite{cite.needed}. A storage, given enough address and data bus width, can represent any combination of logic gates, with the key advantage of being reprogrammable. Thus, nowadays, control units are often not more than a piece of storage and a timing unit. 




\begin{arch-requirement}
  Must produce the clock signal. 
\end{arch-requirement}

Each clock cycle shall be counted to produce a timing state. 
The number of required timing states can only be determined once the exact instruction set is defined. Thus, the space needed in the control word for the timing state can only be estimated for now. I deem three bits not enough, as it would only allow for $2^3$ timing states. 

The required states for a basic jump instruction already are: 
\begin{enumerate}
  \item Increment PC
  \item Fetch instruction
  \item Fetch second data word (instruction argument) and increment program counter by it.
  \item Fetch instruction from new program counter
  \item Execute the new instruction
\end{enumerate}
    
As the execution of the final instruction will probably take more than one step, the total count of instructions is 6-7, which is to close to the maximum allow any flexibility. Thus four bits on the micro instruction word shall be dedicated.

\begin{arch-requirement}
  Must count clock cycles to produce 16 timing states and reset to zero once 16 was reached or the control word indicated to do so.
\end{arch-requirement}

To compensate for any redundant timing states a bit to break out of the current instruction and skip to the next one is needed. 
\begin{feat-requirement}
  Must, given a specific output of the microcode break out of the current instruction (reset state signal).
\end{feat-requirement}



- explain but kinda self explanatory?
\begin{arch-requirement}
  Must produce the control word (micro instruction) from instruction and state.
\end{arch-requirement}

Turing
\begin{turing-requirement}
  Must produce the control word from flags.
\end{turing-requirement}



Given that for the computer to function every instruction must be first pointed to my the program counter and then loaded into the control unit, the relevant control words must be setup in the microcode. 
\begin{feat-requirement}
  The control word generated by the first state, regardless of flag and macro instruction, must always increment the programm counter by one. 
\end{feat-requirement}

\begin{feat-requirement}
  The control word generated by the second state, regardless of flag and macro instruction, must always fetch the current instruction and store it in the macro instruction register.
\end{feat-requirement}


\begin{feat-requirement}
  Must be reprogrammable for every computer run.
\end{feat-requirement}






To let programs end in a way. Halt instruction

\begin{feat-requirement}
  Must halt given micro/macrocode instruction to do so. 
\end{feat-requirement}

\begin{feat-requirement}
  Must reset given micro/macrocode instruction to do so. 
\end{feat-requirement}
// 1 clock cycles
    // 2 state gen
    // 3 state reset
    // 4 microinstruction word
    // 5 control word
    // 6 control word decoding
    
List of requirements not final.


% end subsection Control Unit

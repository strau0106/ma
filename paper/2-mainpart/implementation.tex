\section{Implementation}



The implementation of the simulation is done in the hardware description language Verilog. The architecture shall be implemented in indivdual modules, each module being tested individually.    
\subsection{Timing}
Neither is the timing governed by any leading principles nor does it have a direct impact on the architecture or its performance. It must be established nonetheless to ensure that the computer is able to function properly.

The clock signal, an equally spaced pulse where $t_{high} = t_{low}$, is the only timing signal required. Given signals from the control unit modules shall latch data onto the bus at the rising edge of the clock signal, and the data shall be read at the falling edge.

\subsection{Arithmetic Logic Unit}
Deriving the netlist of the ALU from the requirements is straight forward.

\begin{table}[H]
\begin{tabular}{ccc}
Type& Name & Purpose \\ \hline
I   & Clock & Timing \\
O   & Bus     & Data output         \\
I   & Register A and B & Data input \\
I   & ALU control word & Control \\
O   & Status flag word & Control
\end{tabular}
\caption{}
\label{tab:alu-i/o}
\end{table}

The first idea was to have one wire per possible operation and then a chain of if statements to select the correct operation. However, quickly it became clear that this would result in large amount of unreadable error-prone code, as most possible inputs would not be properly defined.

Finally, I chose another approach. I defined an enum for every possible operation and then used a switch case to apply the correct operation. This does not only greatly increase code readability, allow for easy addition of new operations, but more importantly also result in a simpler hardware. 

Not synthed to with one or gate in the end, bigger impedance on bus more connected to it less stable. 

Finally, output is tri state  

carry flag is calculated by lpadding the addition with one and then check if the first bit is 1. the remaing 8 bits are written on the bus. 

zero flag is zero if 0 (and all regs)


% \lstinputlisting[language=verilog]{../computer/src/modules/alu.sv}

\subsection{Memory}

Memory impl is slightly more convoluted. 

Control signals. 

Which op to perform, which address register to use, and which of the two data words to use. 

And bus in out. 

Two register


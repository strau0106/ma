\section{Architecture}

Given the project goals, the architectures structure, so the architectures indivudal components, units, are largely already given. 

  The requirement for turing completeness, also known as computationally universal, so the requirement to be able to exectue any algorithm \cite{cit.needed}, poses several restrictions. First of all, a computationally universal cannot exist, HALTING PROBLEMB

  SO generally defined as:
  
  \cite{cit.needed}


difference in bus lengths, 8-bit data cannot be infinite can it?
,
The Von-Neumann-Architecture \cite{vonneuman1945a} intends that the architecture is split up into 
alu
cu 
memory 
I/O
bus 

Given the requiremenmt for 8bit and infinite memory, a deviation is required. 

data and addr bus will be split to a certainn extent. in a way the first 8bit of the data bus can be latched onto the addr bus\dots
finally also the control bus will be external, as this is once again much wider. also from the component layout a control bus that is the same as the data bus would still make sense. 

this however does not limit the architecture to still follow the single bus and data word at a time. (which is why the VNA is still promiinent today. no race conditions)

Additionally there is no problem adhering to the VNC. 


Onto bus at rising clock
from bus at falling clock


\subsection{Arithmetic Logic Unit}

\begin{turing-requirement}
  Ability to add two data words.
\end{turing-requirement}

\begin{turing-requirement}
  Generation of status flags (carry, zero and/or remainder of divison). 
\end{turing-requirement}

\begin{arch-requirement}
  Ability to take in data from 2 registers.
\end{arch-requirement}

\begin{arch-requirement}
  Ability to output data to the bus. 
\end{arch-requirement}

\begin{arch-requirement}
  Ability to execute all calculations within one timing state.
\end{arch-requirement}

\begin{feat-requirement}
  Ability to subtract two data words.
\end{feat-requirement}

\begin{feat-requirement}
  Ability to multiply two data words.
\end{feat-requirement}

\begin{feat-requirement}
  Ability to divide two data words.
\end{feat-requirement}

\begin{feat-requirement}
  Ability to shift a data word bitwise.
\end{feat-requirement}

\begin{feat-requirement}
  Ability to rotate a data word bitwise.
\end{feat-requirement}

\begin{feat-requirement}
  Must not exert undefined behaviour in case of undefined control signals. 
\end{feat-requirement}

\begin{feat-requirement}
  Must not output unless control signals indicated to do so. 
\end{feat-requirement}

\begin{feat-requirement}
  Ability to AND, OR and XOR two data words.
\end{feat-requirement}

\begin{feat-requirement}
  Ability to perform NOT on a data word.
\end{feat-requirement}

Each OP as individual wire, because decoding instructions is not the purpose of the ALU, although this means, that it needs to be checked if it is actually only one OP. (This is normal, vgl. Digital Computer electronics p. 175/ von Neumann). 

As per \cite{vonneuman1945a}


Also decreased complexity 9 vs. 4 wires and a bunch of decoding logic? Not limited in package pins etc. 

Direct Register Access because wanting to save timing states and is more easier. 

Given those requirements a list of signals going into and out of the module can be compiled. 

\begin{table}[]
\begin{tabular}{ccc}
Type& Name & Purpose \\ \hline
I   & Clock & Timinga \\
O   & Bus     & Data output         \\
I   & Register A and B & Data input \\
I   & ALU control word & Control \\
O   & Status flag word & Control
\end{tabular}
\caption{}
\label{tab:alu-i/o}
\end{table}

Given this Netlist

Karnaugh map of Operations. 


\lstinputlisting[language=verilog]{../computer/src/modules/alu.sv}

op sum to check if only one op is started

check if out to make sure to only output if said to do so. (three state output)

Each indivdual operation is performed if given.

Only one bit rotations possible since there is no length that can be put in, since only one, the second data word, value can be given in. Additionally, the implementation is much easier, if just one bit...


\subsection{Memory}

\begin{feat-requirement}

\end{feat-requirement}


\subsection{Registers}

\begin{turing-requirement}
  Ability to store a data word.
\end{turing-requirement}

\begin{turing-requirement}
  Ability to output a data word.
\end{turing-requirement}

\begin{turing-requirement}
  Ability to load a data word for storage.
\end{turing-requirement}

\subsection{Control Unit}
Following the VNA the module orchestrating the interaction and operation of all modules is called the Control Unit. It is responsible for the generation of the control word, based on the macro program, the current state of the computer and current timing information. For every given combination of this input the control word must produce a specific output. In historic computer architectures this was achieved by carefully creating a net of logic gates, which would produce the correct output for every input. With the increasing complexity and the fact that once physically produced this control logic could not be changed, the concept of the microcode was invented \cite{cite.needed}. A storage, given enough address and data bus width, can represent any combination of logic gates, with the key advantage of being reprogrammable. Thus, nowadays, control units are often not more than a piece of storage and a timing unit. 




\begin{arch-requirement}
  Must produce the clock signal. 
\end{arch-requirement}

\begin{arch-requirement}
  Must count clock cycles to produce timing states.
\end{arch-requirement}


\begin{feat-requirement}
  Must be reprogrammable for every computer run.
\end{feat-requirement}


\begin{arch-requirement}
  Must produce the control word (micro instruction) from instruction and state.
\end{arch-requirement}




16 makes sense, 8 could under some circumstances not be enough so 4 bit.
As most instructions however, would also work with about 4 states. redundant states shall be skippable.

\begin{feat-requirement}
  Must, given a specific output of the microcode break out of the current instruction (reset state signal).
\end{feat-requirement}


\begin{turing-requirement}
  Must produce the control word from flags.
\end{turing-requirement}

To let programs end in a way. Halt instruction

\begin{feat-requirement}
  Must halt given micro/macrocode instruction to do so. 
\end{feat-requirement}

\begin{feat-requirement}
  Must reset given micro/macrocode instruction to do so. 
\end{feat-requirement}
// 1 clock cycles
    // 2 state gen
    // 3 state reset
    // 4 microinstruction word
    // 5 control word
    // 6 control word decoding
    
List of requirements not final.


% end subsection Control Unit


\input{2-mainpart/architecture/bus}

\input{2-mainpart/architecture/input_output}

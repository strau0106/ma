\section{Architecture}

Given the project goals, the architectures structure, so the architectures indivudal components, units, are largely already given. 

  The requirement for turing completeness, also known as computationally universal, so the requirement to be able to exectue any algorithm \cite{cit.needed}, poses several restrictions. First of all, a computationally universal cannot exist, HALTING PROBLEMB

  SO generally defined as:
  
  \cite{cit.needed}


difference in bus lengths, 8-bit data cannot be infinite can it?
,
The Von-Neumann-Architecture \cite{vonneuman1945a} intends that the architecture is split up into 
alu
cu 
memory 
I/O
bus 

Given the requiremenmt for 8bit and infinite memory, a deviation is required. 

data and addr bus will be split to a certainn extent. in a way the first 8bit of the data bus can be latched onto the addr bus\dots
finally also the control bus will be external, as this is once again much wider. also from the component layout a control bus that is the same as the data bus would still make sense. 

this however does not limit the architecture to still follow the single bus and data word at a time. (which is why the VNA is still promiinent today. no race conditions)

Additionally there is no problem adhering to the VNC. 


Onto bus at rising clock
from bus at falling clock


\subsection{Arithmetic Logic Unit}

\begin{turing-requirement}
  Ability to add two data words.
\end{turing-requirement}

\begin{turing-requirement}
  Generation of at least one status flag (overflow, underflow and/or remainder of divison). 
\end{turing-requirement}

\begin{arch-requirement}
  Ability to take in data from 2 registers.
\end{arch-requirement}

\begin{arch-requirement}
  Ability to output data to the bus. 
\end{arch-requirement}

\begin{arch-requirement}
  Ability to execute all calculations within one timing state.
\end{arch-requirement}

\begin{feat-requirement}
  Ability to subtract two data words.
\end{feat-requirement}

\begin{feat-requirement}
  Ability to multiply two data words.
\end{feat-requirement}

\begin{feat-requirement}
  Ability to divide two data words.
\end{feat-requirement}

\begin{feat-requirement}
  Ability to shift a data word bitwise.
\end{feat-requirement}

\begin{feat-requirement}
  Ability to rotate a data word bitwise.
\end{feat-requirement}

\begin{feat-requirement}
  Must not exert undefined behaviour in case of undefined control signals. 
\end{feat-requirement}

\begin{feat-requirement}
  Must not output unless control signals indicate to do so. 
\end{feat-requirement}

\begin{feat-requirement}
  Ability to AND, OR and XOR two data words.
\end{feat-requirement}

\begin{feat-requirement}
  Ability to perform NOT on a data word.
\end{feat-requirement}


Given those requirements a list of signals going into and out of the module can be compiled. 

\begin{table}[]
\begin{tabular}{ccc}
Type& Name & Purpose \\ \hline
I   & Clock & Timinga \\
O   & Bus     & Data output         \\
I   & Register 1 and 2 & Data input \\
I   & ALU control word & Control \\
O   & Status flag word & Control
\end{tabular}
\caption{}
\label{tab:alu-i/o}
\end{table} 


\subsection{Memory}

\begin{feat-requirement}
Must allow writing to and reading from a memory location
\end{feat-requirement}

\begin{feat-requirement}
Must store, for each memory address, store two data words. 
\end{feat-requirement}

\begin{feat-requirement}
Must retrieve (reword, both retrieve and access) data with an absolute memory address. 
\end{feat-requirement}

\begin{feat-requirement}
Must retrieve data with a relative memory address. 
\end{feat-requirement}

\begin{feat-requirement}
    Must be able to store the program counter and memory address register.
\end{feat-requirement}

\subsection{Registers}

Registers are technically speaking redundant, as the computer already has a form of memory. In practice regs still exist: 
- Regs are in silicon accessible faster than memory. 
- Storage acccess in case of this specific arch is simpler w/o relative addressing etc. 
- ALU needs access to what it calculates, thus registers, unless directly connected to memory, which in practice is not possible (memory seperate silicon from chip). 
Thus registers are considered feature requirements

\begin{feat-requirement}
  Ability to store a data word.
\end{feat-requirement}

\begin{feat-requirement}
  Ability to output a data word.
\end{feat-requirement}

\begin{turing-requirement}
  Ability to load a data word for storage.
\end{turing-requirement}


\subsection{Control Unit}

The Control Unit:
\begin{arch-requirement}
  Must produce the control word from instruction and clock.
\end{arch-requirement}

\begin{turing-requirement}
  Must produce the control word from flags.
\end{turing-requirement}

\begin{feat-requirement}
  Must, given a specific output of the microcode break out of the current instruction. 
\end{feat-requirement}

\begin{feat-requirement}
  Must be reprogrammable for every computer run.
\end{feat-requirement}


% end subsection Control Unit


\subsection{Bus}


\input{2-mainpart/architecture/input_output}

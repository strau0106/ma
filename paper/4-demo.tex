\chapter{Functional Demonstration}
This chapter is intended to demonstrate the function of the computer as per defined in section \ref{goal} by running at first simple well known algorithms \cite{chatgptalgoidea}, and finally demonstrating Turing equivalence by implementing a \textit{Brainfuck} interpreter. % Nope it wasn't chatgpt, it was THC.




\section{Fibonacci Sequence}
The Fibonacci sequence is a simple sequence of natural numbers, where an element in the sequence is given by the sum of the perceding two elements.

The required instructions for generating assembly are thus relatively simple. The only instructions required are those for moving data in and out of registers, initalizing them and adding them. A halt instruction is also added so that the simulation can stop when calculations are done. 

For this demonstration the calculations do not yet happen in a loop but the instructions are just repeated a number of times. Loops and conditions are intended to be demostrated in section \ref{sec:n-num}.

Creating the control words for the instructions is straight forward. For the load instructions, the memory is set to read the second data word, from the current instructions, either storing it into register A or register B. The move instructions set one of the registers to output and one to load. Finally the ADD instruction, activates the ALU, sets its operation to "ADD" and sets the C register to load. 

The final macro programm then looks like this: 
\begin{lstlisting}[language={[x86masm]Assembler}, caption=Assembly code to calculate assembly, label=lst:fib]
LDA 0 ; Initalize register A with 0
LDB 1 ; Initalize register B with 1
ADD ; Add register A and B and store into register C
MBA ; Move register B to A
MCB ; Move register C to B
... ; Repeat ADD, MBA, MCB n times to reach the (n+1)th element
HLT ; Signal execution is complete
\end{lstlisting}

After generating the microcode and macro program, the following content of the three used registers can be observed during the execution of the programm, assuming that the three instructions are repeated 10 times:

\begin{timingdiag}[!ht]
\begin{tikztimingtable}
    macro\_instruction  & 0.58L1.2D{1}1.2D{2}1.2D{3}1.2D{4}1.2D{5}1.2D{3}1.2D{4}1.2D{5}1.2D{3}1.2D{4}1.2D{5}1.2D{3}1.2D{4}1.2D{5}1.2D{3}1.2D{4}1.2D{5}1.2D{3}1.2D{4}1.2D{5}1.2D{3}1.2D{4}1.2D{5}1.2D{3}1.2D{4}1.2D{5}1.2D{3}1.2D{4}1.2D{5}1.2D{3}1.2D{4}1.2D{5}0.2D{6} \\
    reg\_a              & 4.58L7.2D{1}3.6D{2}3.6D{3}3.6D{5}3.6D{8}3.6D{13}3.6D{21}3.6D{34}2.2D{55} \\
    reg\_b              & 2.18L7.2D{1}3.6D{2}3.6D{3}3.6D{5}3.6D{8}3.6D{13}3.6D{21}3.6D{34}3.6D{55}1.0D{89} \\
    reg\_c              & 3.38L3.6D{1}3.6D{2}3.6D{3}3.6D{5}3.6D{8}3.6D{13}3.6D{21}3.6D{34}3.6D{55}3.4D{89} \\
\end{tikztimingtable}
\caption{Execution of Listing \ref{lst:fib}. Signal names adapted for readability.}
\end{timingdiag}

Manually calculation of the $11$th element leads to the same result of $89$.

\section{Sum of first $n$ natural Numbers} \label{sec:nth-sum}
The goal of this demonstration is to show the functioning of the conditional operation and thus also looping. The program shall continously add the next natural number until the sum would be greater than 255 and thus the carry flag triggered. The following python code illustrates the approach and can be used to calculate the expected result.

\begin{lstlisting}[caption=Python code for the generation of the sequence]
sum = 0
num = 1
while (sum+num) < 255:
    sum += num
    num +=1
print(sum)
print(num)
\end{lstlisting}

Executing this script results in $253$ for the expected sum and $23$ as the result for the number of iterations. 

The assembly instructions required for this are limited to initializing the registers, adding and moving around as well as a jump instruction, that is only performed when the carry flag is not set. 

\begin{lstlisting}[caption=Assembly code for the generation of the sum of $n$ natural numbers below 255, label=lst:nsum]
LDB 0
LDC 0 ; Init the registers with 0
LDA 1 ; Load 1 into register A
ADB   ; Add registers A and B and store into B
MCA   ; Move register C into A
ADC   ; Add registers A and B and store into C
JNC 5 ; Jump four instructions backward if carry not set. 
; Jump is instructed to one beforehand so, it points to ADB after the increment
HLT   ; Halt computer
\end{lstlisting}

After configuring the microcode and assembler with these instructions, the sum, that is expected to be $253$ and the last summand $23$ can be observed in the timing diagrams generated during execution: 

\begin{timingdiag}[!ht]
\begin{tikztimingtable}
is\_carry & 55.552L0.84H \\
halt & 56.392LG \\
reg\_a & 1.352L0.96D{1}1.44L3.36D{1}1.44D{3}0.96D{1}1.44D{6}0.96D{1}1.44D{10}0.96D{1}1.44D{15}0.96D{1}1.44D{21}0.96D{1}1.44D{28}0.96D{1}1.44D{36}0.96D{1}1.44D{45}0.96D{1}1.44D{55}0.96D{1}1.44D{66}0.96D{1}1.44D{78}0.96D{1}1.44D{91}0.96D{1}1.44D{105}0.96D{1}1.44D{120}0.96D{1}1.44D{136}0.96D{1}1.44D{153}0.96D{1}1.44D{171}0.96D{1}1.44D{190}0.96D{1}1.44D{210}0.96D{1}1.44D{231}0.96D{1}1.28D{253} \\
reg\_b & 1.832L2.4D{1}2.4D{2}2.4D{3}2.4D{4}2.4D{5}2.4D{6}2.4D{7}2.4D{8}2.4D{9}2.4D{10}2.4D{11}2.4D{12}2.4D{13}2.4D{14}2.4D{15}2.4D{16}2.4D{17}2.4D{18}2.4D{19}2.4D{20}2.4D{21}2.4D{22}1.76D{23} \\
reg\_c & 2.792L2.4D{1}2.4D{3}2.4D{6}2.4D{10}2.4D{15}2.4D{21}2.4D{28}2.4D{36}2.4D{45}2.4D{55}2.4D{66}2.4D{78}2.4D{91}2.4D{105}2.4D{120}2.4D{136}2.4D{153}2.4D{171}2.4D{190}2.4D{210}2.4D{231}2.4D{253}0.8D{20} \\
\end{tikztimingtable}
\caption{Execution of Listing \ref{lst:nsum}. Signal names adapted for readability}
\end{timingdiag}



\section{Brainfuck}
Proving computational universality or Turing completeness can become very convoluted. The simplest way to prove it is to demonstrate Turing equivalence to a Turing machine, by simulating one. Directly simulating a Turing machine is a valid and feasible method to do so. However, Turing machines have the concept of "states" in which the machine can be. As this architectures' concept of states is limited complex additional logic would need to be programmed. A Turing complete language that is more easily interpretable by this architecture is \textit{Brainfuck}. Brainfuck is an esoteric language that generally consists only of six instructions for logic, and two for input and output. It is however not standardized by any governing authority, and thus has a vast number of inscrutable definitions, standards and extensions. Apart from not implementing the input and output instructions (\texttt{.} and \texttt{;}), deviating from, what seems to me the given standard, the loop is implemented as a \texttt{do-while} loop instead of a normal \texttt{while} loop. This significantly eases implementation, as the compiler or interpreter doesn't have to look ahead in the program, but only back. This has no impact on Turing equivalence. 

\begin{table}[H]
\begin{center}
\begin{tabular}{ccc}
Instruction    & Purpose                  & C++ equivalent                       \\ \hline
\texttt{+}             & Increment at the pointer & \texttt{ptr*++;}                   \\
\texttt{-}               & Decrement at the pointer & \texttt{ptr*--;}                   \\
\texttt{\textgreater{}} & Increment the pointer    & \texttt{ptr++;}                    \\
\texttt{\textless{}}   & Decrement the pointer    & \texttt{ptr--;}                    \\
\texttt{{[}}            & Begin loop               & \texttt{do \{}                     \\
\texttt{{]}}            & End loop              & \texttt{\} while(ptr*);}
\end{tabular}
\end{center}
\caption{Brainfuck instruction table}
\label{tab:brainfuck-instructions}
\end{table}

All instructions are logically split up in multiple macro instructions. Both \texttt{+} and \texttt{-} are structured the same way, first the content of the current cell is loaded into register A and register 2 is loaded with one. Then, depending on the instruction, either an addition or a subtraction is performed by the ALU and stored back into the current cell. For \texttt{\textgreater} and \texttt{\textless} a one is once again loaded into the B register and a relative addition or subtraction is preformed on the \texttt{memory\_address\_register}. For the \texttt{{[}} no macro instructions are set. Instead, the current instruction count is stored temporarily. Once the corresponding \texttt{{]}} is reached, the current memory cell is compared against zero. If the cell content is larger than zero a jump backwards is performed. The jump back is calculated with the previously stored instruction count and the current instruction count.

The following instruction set was thus developed:
\begin{table}[]
    \begin{centering}
    \begin{tabular}{cc}
    Instruction & Purpose                                                                  \\ \hline
    LDA         & Load value, pointed to by memory address register, into register A       \\
    LDB         & Load second data word into register B                                    \\
    ADS         & Add register A and B and store into current memory address               \\
    SUS         & Subtract register A and B and store into current memory address          \\
    REA         & Add to the memory address register, what is currently in register B      \\
    RES         & Subtract to the memory address register, what is currently in register B \\
    JNE         & Jump back if not equal, as much as register B specifies                  \\
    HLT         & Halt computer                                                           
    \end{tabular}
\end{centering}
    \caption{Brainfuck macro instruction set.}
    \label{tab:brainfuck-macro-instructions}
    \end{table}

Before any of the Brainfuck program is converted to these macro instructions. The memory address register is first moved away from zero, such that it does not interfere with the program code.

Finally, in a loop the macro instructions for the following program are computed: 

\begin{lstlisting}[caption=Brainfuck program, label=lst:brainfuck]
    ++>>+++++[-[-]]>>+++
\end{lstlisting}

This program first increments the first memory cell by two, then moves two to the right, and increments this by 5, and then preforms two loops. The outer loop only performs one subtraction, before another loop is started that is executed four times, such that the cell is zero, which also causes the second loop to exit. Finally, the pointer is incremented twice again and then the cell is incremented thrice. 

The test bench then checks if the first address is at zero, and if the fourth address is 3.

The program is written arbitrarily, has each instruction once, and has nested loops, to ensure, that nesting functions. 



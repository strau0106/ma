\chapter{Implementing the Simulation}

\section{Development Operations }
Implementation of the DevOps was realised by leveraging the version control system (git) and the GitLab CI/CD platform. Replicating the local environmenta, a Debian based docker container \cite{dockerVerilator} was used in the pipelines. The pipelines is also set up to report code coverage and test results to the version control system platform for easy reference

As the version control system infrastructure does not provide any infrastructure to execute these pipelines, e.g. servers to run the docker containers, seperate machines were put in place and configured.



\section{Build System}
Verilator transforms the Verilog code into a binary and header files to address this binary. This, together with the testbench must be compiled into an executable. A google search of "Verilator GoogleTest" led to a repository containing a demostration for a SystemVerilog, Verilator and GoogleTest toolchain \cite{toolchain}. Luckily this is the exact combination of tools I want to use. 

GNU Make is used to first execute Verilator on all targets, which are defineable in the Makefile. The Makefiles which are generated by Verilator during \textit{verilation} are then set as additional Make targets and the corresponding testbench file is included.

Small adaptions of this buildchain were necessary to make it fit this project. Another folder of Verilog code, the \texttt{package} folder, must be always included, so that packages, in this project containing all enum definitionsm, can be used in the modules. Instead of passing only the Verilog files that are necessary, the build system now passes all source files and sets which module is the highest in the module hierarchy. 

Some flags that allow for Verilators timing and debugging features also have to be enabled.

The testbench example can be used, after adapting all necessary class names for all modules that do not generate, but only use the clock signal. The testbench 
for when Verilator needs to handle time, because a clock signal is generated, will be detailed later.

\section{Arithmetic Logic Unit}
Following the von Neumann Architecture all components recieve (and send) control signals, and are connected to the bus. 

There are two options for the ALU recieving the control signal indicating which operation to perform. Either each operation is represented by a single control signal or each operation is assigned a binary representation and then decoded in the ALU. The first approach would conform more to the VNA as all decoding would be performed in the control Unit, whereas with the second approach, the instruction would only fully be decoded in the ALU. The second approach is however much more efficient and readable. The first approach would require a 10 bit control signal whereas the second approach would only require $ceil(\log_2 10) = 4$ bits. Additionally the operations could be represented easily as an enumeration in Verilog ensuring readability. At first the first option was implemented and tested, but I decided to switch to the second option, because readability was not given. 

The two registers A and B are connected to the ALU directly, to conform with Arch. Req. 2.1.1. To implement testcases, the names of all signals must first be set and a module definition with these ports created, so that the testcases can be written.

\begin{table}[H]
\begin{tabular}{cccc}
  Type& Name & Purpose & \texttt{name}\\ \hline
  I   & Clock & Timing & \texttt{clock}\\
  O   & Bus     & Data output & \texttt{out}        \\
  I   & Register A and B & Data input & \texttt{register1} and \texttt{register2} \\
  I   & ALU control word & Control & \texttt{alu\_op\_e}\\
O   & Flag word & Control & \texttt{alu\_flag\_e}
\end{tabular}
\caption{ALU port list}
\label{tab:alu-i/o}
\end{table}

The implementation of the testcases for the ALU is straightforward for all requirements specifying an arithmetic or logic operation. Generally two arbitrary values are loaded into the registers and the correct control signals are sent to the ALU the clock cycled and then checked if the output is correct. 


\begin{lstlisting}[language=c++]
alu_dut->register1 = 1; 
alu_dut->register2 = 2; // The two registers are initalized with arbitrary values

alu_dut->op = alu::control::alu_op_e::ADD; // The control signal to add the two registers is set 

AdvanceClock();  // Clock is advanced, so that the operation is performed

EXPECT_EQ(alu_dut->result, 3); // Check if the result of the operation is correct.
\end{lstlisting}

The same applies to the tests for flag generation. Values are loaded to produce either of the flags and then it is tested if the flags are correctly generated.

The test case for Feat. Req. \ref{req:alu-no-output} is implemented by assigning arbitrary values and performing an operation without however enabling the output signal, finally checking if any data was latched onto the bus.  

Finally, the Verilog body of the module was based around a single \texttt{always\_ff} block. A switch case executes the correct operation based on \texttt{alu\_flag\_e}. The zero flag is generated by a continous assignment. The carry flag is generated by left padding the addition by one bit, removing the padding before latching the data onto the bus. The flags are latched into a register whenever an operation is performed, such that they are presistent until the next operation.

\section{Memory}

Following the principle of the ALU the memorys opcode is an enum. Apart from the control signal for the operation, the memory module requires additional control signals: the data word selector and address register selector.

All operations, except for absolute memory access requires an indication of which address register, PC or MAR, to use. The read and write operations additionally require the indication of the data word. 

\begin{table}[H]

  \centering
\begin{tabular}{cccc}
 Type & Name               & Datatype                       & name                          \\ \hline
 I    & Bus input          & \texttt{bit{[}7:0{]}}          & \texttt{input}                \\
 O    & Bus output         & \texttt{bit{[}7:0{]}}          & \texttt{output}               \\
 I    & Operation          & \texttt{memory\_op\_e}         & \texttt{op}                   \\
 I    & Data Word Selector & \texttt{bit}                   & \texttt{data\_word\_selector} \\
 I    & Address Register Selector       & \texttt{address\_register\_selector\_e} & \texttt{address\_register\_selector}        \\
 I    & Clock              & \texttt{bit}                   & \texttt{clock}               
 \end{tabular}

 \caption{Memory port list}
 \label{tab:memory-io}
\end{table}

For all tests, involving read operations, a value must be stored directly into the array that represents the physical memory cells in the simulation, this variable must also be defined already. To verify correct calculation of the different address modes the two address registers must also be defined. 

\begin{table}[H]
  \begin{center}
\begin{tabular}{ccc}
    Name               & Datatype                       & name                          \\ \hline
    Cells              & \texttt{bit{[}7:0{]} {[}1-ADDR\_BUS\_WIDTH+1{]}-1} & \texttt{cells} \\
    MAR                & \texttt{bit{[}(ADDR\_BUS \_WIDTH-1:0{]}} & \texttt{memory\_address\_register} \\
    PC                & \texttt{bit{[}(ADDR\_BUS\_WIDTH-1:0{]}} & \texttt{program\_counter} \\
    \end{tabular}
  \end{center}
    \caption{Memory internal net list}
    \label{tab:memory-internal-nets}

    
   \end{table}
   

Once again with the port and net list the test cases for the memory module are implemented. For the read operations and then the correct address is either loaded directly (absolute reads) or is the result of several operations (relative reads) where the address must first be calculated. For write operations the address is also either loaded directly or composed of multiple operations. The value that is to be wrriten is written onto the bus and the write operation is performed. The array is then checked at the address to see if the value is correctly written.

The read and write operations are implemented by reading/writing to the cells array with the index being the selected address register with the data word selector being appended as the least significant bit. The selected address bus is generated by a continous assignment based on the control signals. 

When specified, the value from the memory cells at the address specified by the selected bus and data word selector is latched onto the bus. When specified, the value on the input bus is written to the memory cells at the address specified by the selected bus and data word selector. They are thus executed in to seperate \texttt{always\_ff} blocks.

As continous assignments are used for the selected address bus, operations on the address registers cannot be done on the selected address bus variable. Instead the address registers are modified directly with an \texttt{if} statement.


\section{Register}
The register will is implemented in two seperate modules, one complying with all requirements, the second one without \ref{req:register-direct-access} direct access for the alu. As they are four registers and only two are connected to the ALU, having direct access present in the other two would be redundant. The register implementing the direct access is named \texttt{reg\_acc}, the other \texttt{reg\_tmp}.

The register module is the simplest of all modules. Only bus input and output, clock and a control signal are required. The bus is connected as input and output.

\begin{table}[H]
  
  \begin{center}
  \begin{tabular}{cccc}
   Type & Name               & Datatype                       & name                          \\ \hline
   I    & Bus input          & \texttt{bit{[}7:0{]}}          & \texttt{input}                \\
   O    & Bus output         & \texttt{bit{[}7:0{]}}          & \texttt{output}               \\
   I    & Operation          & \texttt{reg\_op\_e}         & \texttt{op}                   \\
   I    & Clock              & \texttt{bit}                   & \texttt{clock}               \\
   O    & ALU Direct Access               & \texttt{bit{[}7:0{]}}          & \texttt{reg\_direct}               \\
   \end{tabular}
  \end{center}
   \caption{Register port list}
   \label{tab:reg-io}
\end{table}

The last entry is only present in the \texttt{reg\_acc} module. The test cases for the register module are implemented similarly to those for the ALU and memory. 

For both modules, the module is split into two \texttt{always\_ff} blocks, one for reading and one for writing. The reading block is executed on the rising edge of the clock signal and the writing block on the falling edge. The reading block is a simple assignment of the output to the input. The writing block is a simple assignment of the input to the output. The direct access is implemented by a continous assignment.


\section{Control Unit}
The Control Units port list can be derived from all the signals that are required on the other modules. 

\begin{table}[H]
  
  \begin{center}
  \begin{tabular}{cccc}
   Type & Name               & Datatype                       & name                          \\ \hline
   I    & Bus input          & \texttt{bit{[}7:0{]}}          & \texttt{input}                \\
   O    & Clock              & \texttt{bit}                   & \texttt{clock}               \\
   O    & Memory Operation   & \texttt{memory\_op\_e}         & \texttt{memory\_op}           \\
   I    & ALU flag          & \texttt{alu\_flag\_e}          & \texttt{alu\_flag}            \\
   O    & ALU control word & \texttt{alu\_op\_e}         & \texttt{alu\_op}                   \\
   I    & 4x Reg Operation   & \texttt{reg\_op\_e}         & \texttt{r\*x\_op}                   \\
    \end{tabular}
  \end{center}
   \caption{Control Unit port list}
   \label{tab:reg-io}
\end{table}

To implement testcases for the control unit additonal variabls internal to the control unit must be defined. To test generation of timing information, the timing state count variable must be defined. The test cases for microcode generation must have access to the microcode storage.

\begin{table}[H]
  \begin{center}
  \begin{tabular}{ccc}
    Name               & Datatype                       & name                          \\ \hline
    Timing state count & \texttt{bit{[}3:0{]}}          & \texttt{state}                \\
    Microcode          & \texttt{bit{[}(\`CW\_WIDTH\-1):0{]} {[}(1$<<$MIW\_WIDTH)\-1{]}} & \texttt{microcode}                \\
  \end{tabular}

\end{center}
  \caption{Control Unit internal variables that are required for testing}
   \label{tab:reg-io}

  \end{table}

All test related to timing are implemented by advancing time in the simulator and observing clock and timing state behaviour, including the state rolling over to zero. The test cases for microcode generation are implemented by loading single control words, generated by a helper library, into the microcode storage at address that are expected, then advancing time and checking if the correct control word is output and if all parts of the control word, generated by the helper library, are output to the correct output signals. 

At every rising edge of the clock signal, the state is counted up, rolling back to 0 once the maximum number of states was reached or the control signal for the next instruction is high.


The microcode is read and written out to the control word at the falling edge of the clock. 

\section{Build system switch}
The original build system turned out to be extremely limited as soon as additional libraries needed to be included. I thus decided to switch away from Make to CMake because I had previous experience with it.

As the build system is not necessarily part of the product, EXPAND, I decided to employ ChatGPT to write the new build system. At first, ChatGPT incorrecctly used Verilator inside of the CMake file, instead of using the Verilator plugin, it called Verilator as a binary. After providing it with Verilator's documentation, ChatGPT fixed the mistake and provided a functioning build system. 

Some additional modifications were made, to correctly include and fetch the GoogleTest library, such that it can be compiled against. 

The new build system does exactly the same as the old one apart from the allowing for easier inclusion and compilation of libraries. Instead of having to provide each indivdual C++ file to the make command, all files in the \texttt{lib} directory are now usable in the testbenches.

\section{Helper Libraries}
The simulation shall be programmable in assembly as per the goals of this project. The architecture, and the simulation for that matter, however are only directly programmable by macro instructions that are in return translated into micro instructions by the control unit. To bridge that gap, several libraries were introduced that make this possible. 

Together these libraries handle the process from instruction definitions and assembly code to microcode and macro programm. The whole process is logically split up into 7 libraries. All libraries implement a builder pattern. 

The process of reaching a functioning programm begins with the definition of the microcode handler. The microcode handler is then loaded with macro instructions. During the intalization of the microcode handler, the states to fetch and load the new instructions are already put into the microcode.

\begin{lstlisting}[caption=Inialization of the Microcode]
auto FibMicrocode = (new Microcode()); // Initalize the Microcode handler

FibMicrocode->AddMacroInstruction( 
    (new MacroInstruction("LDA")) // Initalize new macro instructions with name LDA. 
        ->set_next_state((new TimingState( // Set first start. 
            (new ControlWord()) 
                ->set_data_word_selector(1)
                ->set_memory_bus_selector(cpu_control::memory_bus_selector_e::PC)
                ->set_memory_op(cpu_control::memory_op_e::READ)
                ->set_rax_op(cpu_control::reg_op_e::LOAD))))); // Set control word bits.
\end{lstlisting}

The control word provided at construction of a \texttt{TimingState} is implicitly set for all possible flags. The \texttt{TimingState} class provides a method to overwrite a control word for a single flag. Jump instructions can harness this to only execute jumps if specific flags are set.

After adding all macro instructions, the microcode is computed. Each macro instruction is given an opcode, which is stored for later reference, remaining states filled with a control word that indicates jumping to the next instruction. Finally, all instructions are converted to binary and concatenated, to be stored in an instance of the simulation. 

To then program in assembly, an assembler must be created and supplied with the opcodes and instruction names. To reduce unncessary complexity by introducing a string parsing, this assembler does not use a classical assembly file but instead the program can be built step by step with a builder pattern. 
\begin{lstlisting}[caption=Inialization of Assembler and programming of macrocode ]
// The assembler is initalized and the Microcode passed to get opcodes
auto FibAssembler = new Assembler(FibMicrocode); 

FibAssembler->next("LDA", 0)->next("LDB", 1); // Add instructions with arguments
for (int i = 0; i < 10; i++) {
    // Add instructions to calculate fibonacci
    FibAssembler->next("ADD")->next("MBA")->next("MCB"); 
}
FibAssembler->next("HLT"); // Finally halt the computer

// Assemble the code and copy into simulation instance.
FibAssembler->StoreIntoModel(cpu_dut->rootp->cpu__DOT__memory__DOT__cells.m_storage); 
\end{lstlisting}

At first, these libraries did not function correctly. Both binaries, microcode and macro program, were seemingly random. After some investigation, the root cause turned out to be unallocated memory. In an attempt to increase the readability I forgot allocating all of the different objects. During one late programming session I was under the impression that the allocation would be performed by the constructor and omitted the new. \texttt{ControlWord()->set\_data\_word\_selector()} is in fact much more readable, than\texttt{{new ControlWord()}->set\_data\_word\_selector()}, however sadly not functions correctly.

A test case was introduced, to ensure, that during programming of the helper libraries, no errors were introduced in the binary arithmetics for control word and microcode generation.

\section{Integration of all modules}
The integration of all modules is achieved by instantiaing all modules and connecting the ports. To test the integrated \texttt{CPU}, refered to as a \textit{unit}, whole macro programs are loaded. This was straightforward. 

After initial tests however, the computer did not function. During close analysis of the timing diagrams created during the tests, it became clear that the macro instruction was not correctly loaded into the instruction register in the control unit. 

\subsection{Timing Issues}
After the first instruction in the assembly is read from the memory at state 0x3, the load signal goes high, which should trigger a latch into the register, given the module code. 

\begin{lstlisting}[caption=Instruction Register Code]
always_ff @(negedge clock) begin
[...]
if (load) begin
  macro_instruction <= bus;
end
[...]
end
\end{lstlisting}

A timing diagram however clearly shows, that the register remains empty.
\begin{timingdiag}[!ht]
\begin{center}
\begin{tikztimingtable}
    clock              & L H L H L H \\ 
    load               & L L H H L L \\
    memory\_op          & 2L 2D{READ} 2L \\ 
    bus[7:0]           & 2D{00} 2D{01} 2D{00} \\
    macro\_instruction[7:0] & 6L \\ 
\end{tikztimingtable}
\end{center}
\caption{Execution of primer micro instructions.}
\end{timingdiag}


Revisiting the corresponding Verilog code, it indicated no reason, on why the latching into the register wouldn't function as long as there is data on the bus. After several tests and some research into assignments in Verilog, it became obvious that there was a architectural flaw in the implementation. But it was unclear where this flaw was. Moving focus from development onto the paper, the flaw immediately clear. As initally intended and written in early drafts of the paper, cf. Git commit 57c6dc669, content should be "Onto bus at rising clock, from bus at falling clock.". This was however never translated into a formal requirement, and thus forgotten during implementation. 

Up until now all operations the computer performed, apart from the control unit, were performed at the rising edge of the clock. This meant that during writing to the bus by one module, another already tried to read that data, but that data hadn't yet propagated to it, which lead to no data being read at all. To rectify this issue all read operations in modules, so control unit, memory, and registers, were moved out of the rising edge \texttt{always\_ff} block into a new \texttt{always\_ff}. 

\begin{arch-requirement} \label{req:bus-read-write}
  All writes to the bus must occur at the rising edge of the clock. All reads from the bus must occur at falling edge of the clock.
\end{arch-requirement}

After an additional bug fix in the microcode generation, where macro program code was overwriting the fetch control word in the final microcode, execution of simple programs was successful.


\section{Conceptional Issues with Jumps}
During development of the demonstration for \ref{sec:nth-sum} it became clear that jumps were not possible as initally intended. 

Jump instructions were intended to work by giving an instruction that has a specific condition in microcode and that has the instruction to jump to as an argument. Given the current implementation of the memory module this would not be possible, as the memory would need to receive two instructions concurrently. 

The first to output the argument, so the second data word, to the bus, the other to subtract or add this data word to program counter. To solve this concurrency issue, either the addressing module receives its own independent control signal, or the argument is instead of loaded with the jump instruction, stored in a register and accessed from there. 

Although the second option would result in a simpler module, and smaller control bus width, the resulting processing overhead would be enormous. As explained earlier memory reads are extremely time expensive and having to read the address offset every time even though it is maybe not needed, warrants this extra complexity.

To implement this change, modifications were needed in the control unit, adding another control signal, the memory module, implementing this second control word, by adding case statements and the helper libraries, so corresponding micro and macro code can be generated. During modification for this new feature it was also noticed that Arch. Req. \ref{req:bus-read-write} was improperly implemented in the memory module, where modifications to the address buses would be handled at rising edge, and not at falling edge. This was mistake was also corrected.

\section{Flag latching}
Further development of the demonstration for \ref{sec:nth-sum} made a mistake in the status flag implementation apparent. Up until now the status flags were generated by a continous assignment in the ALU, which depended on the enable signal of the ALU. This lead to the flags never being available to the control unit, as the continous assignment returned to zero after disabling the ALU. Moving this assignment to latch into the output register rectifies this problem.


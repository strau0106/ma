\subsection{Verilog Crash Course}
Verilog files normally contain just a single module. A module is declared with the \texttt{module} keyword. To complete the declaration the module must be named and ports specified. Ports are the connections going in and out of a module. They can either be an \texttt{input}, an \texttt{output} or bidirectional, denoted by \texttt{inout}. Ports are then specified like any other variable, by specifying the datatype, length and name. For this paper usage of datatypes shall be limited to, \texttt{wire}, \texttt{reg} and for readability reasons \texttt{enum}.  

different assignments
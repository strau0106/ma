% Paket um vordefinierte Texte (z.B. "Inhaltsverzeichnis") auf Deutsch zu übersetzen
\usepackage[english]{babel}

% Paket um Schriftarten festzulegen (für XeLaTeX)
\usepackage{fontspec}

% serifenfreie Schriftart Arial festlegen
% \setsansfont{}ð

% serifenfreie Schriftvariante verwenden
\renewcommand{\familydefault}{\sfdefault}

% Paket um Grafiken (JPG, PNG, PDF) einzubinden
\usepackage{graphicx}

% Paket für Zeilenabstand
\usepackage{setspace}

% Paket für korrekte Anführungszeichen
\usepackage{csquotes}

% Paket für selbst definierte Kopf- und Fusszeilen
\usepackage{scrlayer-scrpage}

% Pakte für Zitate und Bibliografie
\usepackage{biblatex}

\addbibresource{literature.bib}

% Paket zum Erzeugen von Platzhaltertext
\usepackage{lipsum}

% Code below courtesy of chatgpt-4o on 2024-06-25

% Package for changing theorem style
\usepackage{amsthm}

% Code to change theorem styling
\setlength{\topsep}{0.1em}
\setlength{\partopsep}{0.1em} % Extra space above theorem if at the beginning of a list
\setlength{\parsep}{0.1em} % Space between paragraphs within the theorem

% End of code courtesy of chatgpt-4o

% Packge for code highlighting
\usepackage{listings}


\usepackage{courier}        % For monospace font
\usepackage{xcolor}         % For background color

\lstdefinestyle{customverilog}{
    language=verilog,                    % Set language to Verilog/SystemVerilog
    basicstyle=\ttfamily\scriptsize,                % Monospace font
    breaklines=true,                     % Break long lines if necessary
    numbers=left,                        % Line numbers on the left
    numberstyle=\tiny,                   % Line numbers in tiny font
    backgroundcolor=\color{gray!10},     % Light gray background
}


\lstset{style=customverilog}


% Package for hyperlinks
\usepackage[hidelinks]{hyperref}

% Package for float placement
\usepackage{float}

% noitemsep
\usepackage{enumitem}
\setlist{nolistsep}
\setlist{noitemsep}

% svg 
\usepackage{svg}

% tikz
\usepackage{tikz}
\usetikzlibrary{shapes.geometric, arrows.meta}

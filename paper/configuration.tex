% Paket um vordefinierte Texte (z.B. "Inhaltsverzeichnis") auf Deutsch zu übersetzen
\usepackage[english]{babel}

% Paket um Schriftarten festzulegen (für XeLaTeX)
\usepackage{fontspec}

% serifenfreie Schriftart Arial festlegen
% \setsansfont{}

% serifenfreie Schriftvariante verwenden
\renewcommand{\familydefault}{\sfdefault}

% Paket um Grafiken (JPG, PNG, PDF) einzubinden
\usepackage{graphicx}

% Paket für Zeilenabstand
\usepackage{setspace}

% Paket für korrekte Anführungszeichen
\usepackage{csquotes}

% Package for dirtree
\usepackage{dirtree}

% Paket für selbst definierte Kopf- und Fusszeilen
\usepackage{scrlayer-scrpage}

% Pakte für Zitate und Bibliografie
\usepackage{biblatex}

\addbibresource{literature.bib}

% Paket zum Erzeugen von Platzhaltertext
\usepackage{lipsum}

% Code below courtesy of chatgpt-4o on 2024-06-25

% Package for changing theorem style
\usepackage{amsthm}

% Code to change theorem styling
\setlength{\partopsep}{1em} % Extra space above theorem if at the beginning of a list
\setlength{\parsep}{0.25em} % Space between paragraphs within the theorem

% End of code courtesy of chatgpt-4o

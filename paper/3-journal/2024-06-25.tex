\section{2024-06-25} % (fold)
\label{sec:2024-06-25}
Today is writing day, wohoo!

I honestly have no clue, how to start writing this stuff\dots

Künstlerische (z.B. Komposition, Inszenierung) oder technologische (z.B. Smartphoneprogrammie-
rung) Arbeiten sind möglich und bestehen aus einem Produkt und einem schriftlichen Kommentar,
der mindestens die drei folgenden Teile umfasst:
̶ eine Beschreibung der Produktidee sowie der für deren Realisierung benötigten - vorhandenen
oder zu erarbeitenden – theoretischen und praktischen Grundlagen (Kenntnisse und Fertigkei-
ten)
̶ eine Prozessdokumentation
̶ eine Analyse und Bewertung des Produkts entsprechend der jeweiligen Wissenschaft

As far as I can recall, my proc doc will be what things i do in my architecture. 

The light in this room is really annoying.

lets quickly write glab cicd for tex paper.

great success. after writing my own docker containers deploying to ghcr and so on. bloodly texlive



What I am thinking of doing rn is what ever

I am really struggling with this. What should I do. 

actually lets follow digital computer electronics principle for explaining an arch. 


So a nice sunnesite lecture later, this is what digital computer electronics does to explain the architecture

architecture:
(every component)
- bus
- pc
- reg
- alu
- control register 

- Instruction set
- Programming

I guess it would be interesting to include all code in the paper explaining it. 

one could also add all of the I/O etc into the paper
basically making it first an implementation instruction, aka requirements, making test cases for each. (first subchapter of say alu) and then describe the process of implementation. 

Testcases would be numbered and then reflected in the respective cpp code. 

10 am coffee break!


% section 2024-06-25 (end)
